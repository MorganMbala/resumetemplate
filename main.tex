\documentclass[letterpaper,11pt]{article}

\usepackage{latexsym}
\usepackage[empty]{fullpage}
\usepackage{titlesec}
\usepackage{marvosym}
\usepackage[usenames,dvipsnames]{color}
\usepackage{verbatim}
\usepackage{enumitem}
\usepackage[hidelinks]{hyperref}
\usepackage{fancyhdr}
\usepackage[english]{babel}
\usepackage{tabularx}
\usepackage{fontawesome5}
\usepackage{multicol}
\setlength{\multicolsep}{-3.0pt}
\setlength{\columnsep}{-1pt}
\input{glyphtounicode}
\usepackage[margin=1.4cm]{geometry}


\pagestyle{fancy}
\fancyhf{} % clear all header and footer fields
\fancyfoot{}
\renewcommand{\headrulewidth}{0pt}
\renewcommand{\footrulewidth}{0pt}

% Adjust margins
\addtolength{\oddsidemargin}{-0.15in}
 \addtolength{\textwidth}{0.3in}

\urlstyle{same}

\raggedbottom
\raggedright
\setlength{\tabcolsep}{0in}

% Sections formatting
\titleformat{\section}{
  \vspace{-4pt}\scshape\raggedright\large\bfseries
}{}{0em}{}[\color{black}\titlerule \vspace{-5pt}]

% Ensure that generate pdf is machine readable/ATS parsable
\pdfgentounicode=1

%-------------------------
% Custom commands
\newcommand{\resumeItem}[1]{
  \item\small{
    {#1 \vspace{0pt}}
  }
}

\newcommand{\classesList}[4]{
    \item\small{
        {#1 #2 #3 #4 \vspace{-2pt}}
  }
}

\newcommand{\resumeSubheading}[4]{
  \vspace{-2pt}\item
    \begin{tabular*}{1.0\textwidth}[t]{l@{\extracolsep{\fill}}r}
      \textbf{#1} & \textbf{\small #2} \\
      \textit{\small#3} & \textit{\small #4} \\
    \end{tabular*}\vspace{-7pt}
}

\newcommand{\resumeSubSubheading}[2]{
    \item
    \begin{tabular*}{0.97\textwidth}{l@{\extracolsep{\fill}}r}
      \textit{\small#1} & \textit{\small #2} \\
    \end{tabular*}\vspace{-7pt}
}

\newcommand{\resumeProjectHeading}[2]{
    \item
    \begin{tabular*}{1.001\textwidth}{l@{\extracolsep{\fill}}r}
      \small#1 & \textbf{\small #2}\\
    \end{tabular*}\vspace{-7pt}
}

\newcommand{\resumeSubItem}[1]{\resumeItem{#1}\vspace{-4pt}}

\renewcommand\labelitemi{$\vcenter{\hbox{\tiny$\bullet$}}$}
\renewcommand\labelitemii{$\vcenter{\hbox{\tiny$\bullet$}}$}

\newcommand{\resumeSubHeadingListStart}{\begin{itemize}[leftmargin=0.0in, label={}]}
\newcommand{\resumeSubHeadingListEnd}{\end{itemize}}\vspace{0pt}
\newcommand{\resumeItemListStart}{\begin{itemize}}
\newcommand{\resumeItemListEnd}{\end{itemize}\vspace{-5pt}}



\begin{document}

%----------HEADING----------
\begin{center}
    {\Large \scshape Morgan Ilunga Mbala} \\[2mm]
    \small Développeur Backend .NET, Azure, SQL Server \\[1mm]
    \footnotesize 
    \faPhone\ \underline{(514) 576-4832} ~ 
    {\faEnvelope\  \underline{morganmbala@icloud.com}} ~ 
    {\faLinkedin\ \underline{\href{https://www.linkedin.com/in/morgan-mbala-6274a1395/}{linkedin.com/in/morgan-mbala-6274a1395}}}  ~
    {\faGithub\ \underline{\href{https://github.com/MorganMbala}{github.com/MorganMbala}}} ~
    % {\faBriefcase\ \underline{\href{https://mscott-portfolio.vercel.app/}{mscott-portfolio.vercel.app/}}}
    \vspace{-8pt}
\end{center}

%-----------EDUCATION-----------
\section{Education}
  \resumeSubHeadingListStart
    \resumeSubheading
      {Institut Teccart}{Diplômé : Septembre 2025}
      {Diplôme d'études collégiales (DEC) en Programmation Web, Mobile et Intelligence Artificielle}
      {Montréal, QC}
    \resumeSubheading
      {Université Protestante au Congo}{Diplômé : Novembre 2022}
      {Baccalauréat en Administration des Affaires et Sciences Économiques}
      {Kinshasa, RDC}
  \resumeSubHeadingListEnd
  
%-----------Experience---------------
\section{Expérience professionnelle}
    \resumeSubHeadingListStart
                \resumeSubheading{Zyrasoft Inc.}{November 2024 - Présent}{Fondateur et Développeur principal}{Montréal, QC} 
                \resumeItemListStart
                  \resumeItem{Fondé Zyrasoft pour répondre au besoin croissant des PME locales d’avoir des solutions logicielles cloud fiables et abordables.}
                  \resumeItem{Conception et déploiement d’applications web sur \textbf{Microsoft Azure} avec intégration de pipelines \textbf{CI/CD GitHub Actions}, réduisant de \textbf{40\% le temps de mise en production}.}
                  \resumeItem{Développement d’API REST sécurisées avec \textbf{ASP.NET Core MVC} et \textbf{SQL Server}, permettant d’automatiser des tâches internes et d’améliorer la performance des requêtes de \textbf{30\%}.}
                  \resumeItem{Mise en place de \textbf{Docker} pour la conteneurisation et le déploiement reproductible, assurant une fiabilité de \textbf{99.9\%} sur les environnements clients.}
                  \resumeItem{Pilotage complet du développement de produits clients, avec intégration d’éléments d’\textbf{intelligence artificielle} (chatbots, reconnaissance faciale) afin d’augmenter l’engagement utilisateur de \textbf{+25\%}.}
                \resumeItemListEnd
            \resumeSubheading{Projet académique : Le Mini-Restaurant}{Été 2024}{Développeur principal}{Montréal, QC} 
                \resumeItemListStart
                    \resumeItem{Projet visant à concevoir une application de gestion de commandes intuitive pour un restaurant fictif, afin de simuler un système de caisse automatisé.}
                    \resumeItem{Développement en \textbf{C\#} avec \textbf{ASP.NET Web Forms}, incluant la logique de calcul du total, les réductions selon le profil client (adulte/enfant/ainé) et le calcul des taxes.}
                    \resumeItem{Résultat : interface fluide et interactive, réduisant le nombre d’erreurs de saisie de \textbf{80\%} lors des tests utilisateurs et augmentant la satisfaction perçue de \textbf{90\%}.}
                  \resumeItemListEnd
            \resumeSubheading{Projet .NET : Charmes du Bois}{Automne 2023}{Développeur principal}{Montréal, QC}
                \resumeItemListStart
                      \resumeItem{Création d’un site e-commerce complet pour un artisan ébéniste en Normandie, visant à digitaliser la présentation et la vente de ses produits.}
                      \resumeItem{Développement sous \textbf{ASP.NET Core MVC} et \textbf{Entity Framework Core}, avec gestion des utilisateurs, du panier et des commandes.}
                      \resumeItem{Implémentation d’un système de paiement via \textbf{API PayPal Sandbox} et déploiement optimisé sous \textbf{Visual Studio 2022}.}
                      \resumeItem{Amélioration du temps de chargement des pages de \textbf{35\%} et fiabilité des transactions atteignant \textbf{100\%} en environnement de test.}
                    \resumeItemListEnd
    \resumeSubHeadingListEnd
    \vspace{-12pt}

%-----------PROJECTS-----------
\section{Projets complémentaires} 
    \vspace{-5pt}
    \resumeSubHeadingListStart
        \resumeProjectHeading
            {\textbf{Site web officiel d’un artiste musicien Zyrasoft Inc.} $|$ \emph{\href{https://brokenskafans.ca}{Website} $|$ \href{https://github.com/MorganMbala/Broken}{Source Code}}}{ $|$ ReactJS $|$ Azure $|$ GitHub Actions}
        \resumeItemListStart
            \resumeItem{Développement complet du site web promotionnel d’un artiste musicien, visant à centraliser son image, ses œuvres et ses événements.}
            \resumeItem{Conception de l’architecture front-end avec intégration du contenu dynamique.}
            \resumeItem{Déploiement sur \textbf{Microsoft Azure} à l’aide de pipelines \textbf{CI/CD GitHub Actions} pour automatiser les tests et la mise à jour en production.}
            \resumeItem{Enregistrement et configuration du nom de domaine personnalisé \textbf{brokenskafans.ca}.}
            \resumeItem{Résultat : livraison d’un produit professionnel en moins de 3 jours, avec un taux de satisfaction client de \textbf{100\%}.}
        \resumeItemListEnd
        \vspace{-10pt}
        \resumeProjectHeading
            {\textbf{Reconnaissance faciale et chatbot IA  Zyrasoft Inc.} $|$ \emph{\href{https://github.com/MorganMbala/IA_RF_API/}{Source Code}}}{ $|$ Numpy $|$ Insightface $|$ Python}
        \resumeItemListStart
            \resumeItem{Développement d’une API de reconnaissance faciale et d’un chatbot IA pour automatiser l’accueil et l’authentification sur des plateformes web.}
            \resumeItem{Utilisation de \textbf{Numpy} et \textbf{insightface} pour le traitement d’images et l’identification biométrique.}
            \resumeItem{Déploiement sur serveur cloud avec documentation technique et tests unitaires.}
        \resumeItemListEnd
    \resumeSubHeadingListEnd
    \vspace{-20pt}

%-----------PROGRAMMING SKILLS-----------
\section{Compétences techniques}
 \begin{itemize}[leftmargin=0.15in, label={}]
    \small{\item{   
     \textbf{Langages}{: Python, Java, SQL, HTML5, CSS, JavaScript, TypeScript, C\#} \\[1mm]
     \textbf{Outils développeur}{: Azure, Postman, Visual Studio, Git, GitHub, Docker, Jira, Kubernetes} \\[1mm]
     \textbf{Frameworks/Bibliothèques }{: ASP.NET Core, Entity Framework, ReactJS, NextJS, VueJS, NodeJS} \\ [1mm]
    }}
 \end{itemize}
 \vspace{-16pt}
 \vspace{3pt}
\vspace{10pt}

\vspace{-15pt}



\end{document}
